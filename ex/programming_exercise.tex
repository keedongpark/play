% !TEX encoding = UTF-8 Unicode

\documentclass{article}

\usepackage{amsmath,amssymb, amsfonts, amsthm, titling, url, array}
\usepackage[hangul]{kotex}
\usepackage{kotex-logo}

\usepackage{iftex}
\ifPDFTeX
  \usepackage{dhucs-nanumfont}
\else\ifXeTeX
  \setmainhangulfont[Ligatures=TeX]{HCR Batang LVT}
  \setsanshangulfont[Ligatures=TeX]{HCR Dotum LVT}
\else\ifLuaTeX
  \setmainhangulfont[Ligatures=TeX]{HCR Batang LVT}
  \setsanshangulfont[Ligatures=TeX]{HCR Dotum LVT}
\fi\fi\fi
\usepackage{tikz}
\usepackage{tcolorbox}

\setlength{\parindent}{0.4em}
\setlength{\parskip}{0.5em}

\theoremstyle{plain}
\newtheorem{thm}{Theorem}[section]
\newtheorem{ex}{Ex}[section]
\newtheorem{lem}[thm]{Lemma}
\newtheorem{prop}[thm]{Proposition}
\newtheorem*{cor}{Corollary}

\theoremstyle{definition}
\newtheorem{defn}{Definition}[section]
\newtheorem{conj}{Conjecture}[section]
\newtheorem{exmp}{Example}[section]

\theoremstyle{remark}
\newtheorem*{rem}{Remark}
\newtheorem*{note}{Note}


\begin{document}

\title{Programming Exercises}
\author{Young Hand}
\date{\today}

\maketitle

\newpage
\tableofcontents
\newpage
\pagenumbering{arabic}


\section{언어}

\paragraph{utf8, ucs16, mbcs간의 코드 변환과 인코딩 사용 방법}
다양한 인코딩이 있고 *nx 계열은 utf8로 정착되었고 윈도우 계열은 ucs16으로 정리되었다. 
utf8을 쓸 때 std::string을 사용하면 되는 유닉스 계열과 ucs16은 wstring을 쓰고 
std::string은 ascii나 mbcs 문자열인 윈도우 때문에 정리가 필요하다. 

\begin{ex}[코드 변환]
ucs16과 utf8 간의 코드 변환 구조를 설명하고 이를 처리하는 함수를 작성하거나 
기존에 있는 코드를 분석하여 정리한다.
\end{ex}


\end{document}
